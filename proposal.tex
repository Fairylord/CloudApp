\documentclass[12pt,a4paper]{article}
\usepackage{indentfirst}
\usepackage{CJK}


\begin{document}
\begin{CJK}{UTF8}{gbsn}

\begin{titlepage}
\begin{center}
\textsc{\Huge 云计算应用开发课程}\\[0.5cm]
\textsc{\LARGE 应用项目计划书}\\[2.5cm]

\begin{tabular}{ |l|l| }
	\hline
	项目名称: & 基于云计算的自助旅游详细方案订制\\ \hline
	队长姓名: & 龚睿奇\\ \hline
	队长电话: & 13631239412 {\slash} 659412\\
	\hline
\end{tabular}
\end{center}
\vspace{10cm}
\begin{center}
\begin{tabular}{ |c|c|c| }
	\hline
		 & 姓名   & 学号	\\ \cline{2-3}
	团队成员 & 龚睿奇 & 12353048	\\ \cline{2-3}
		 & 乔丹   & 12353???	\\ \cline{2-3}
		 & 杨奇标 & 12353???	\\
	\hline
\end{tabular}
\end{center}

\end{titlepage}

\section{应用介绍}
	% section 1.1
	\subsection{背景介绍}
	如今,人们的精神生活日趋丰富,逢年过节外出旅游是常有的事情。对于学生群体及上班一族而言,趁着周末在城市的周边或郊外进行短途旅行,亦很常见。随着人们对旅行的要求不断提高,由传统旅行社提供的传统意义上的“跟团“旅游服务,其缺点和不足日趋显现,已经不能满足一部分人的需求。因此,”自助游”这种旅游方式,即由自己来定制旅行的全过程,包括要去的景点,日程的安排,交通方式的选择,食宿的安排等应运而生。“自助游”的这种个性化,全定制的旅行方式十分地诱人,因此也赢得了广泛的欢迎。但是其缺点也很突出,概括而言就是一个词---“麻烦”:一切的饮食、住宿、交通等细节,都需要由自己来安排。这对一些没有接触过“自助游”的人来说,还是相当有门槛的。人们不仅仅要消耗额外的时间和精力去规划行程,而且在规划行程的过程中,很可能使用了错误或过时的信息,造成规划失误,以至于最终旅途不顺,甚至造成财产损失。正是“自助游”的这些缺点所带来的风险和不确定性,导致还有相当一部分人对其心存芥蒂。

	经过讨论和分析,我们发现,人们在策划旅行的时候,真正需要去关心的,仅仅是那些要参观的景点,以及那些要进行的旅行体验;而至于除此以外的其他方面,例如交通,饮食及住宿等细节\footnotemark,都不属于旅行真正有意义的部分。而当一个用户选择了“自助游”,并且在规划自己的旅游行程时,他关心得更多的,却是这些不重要的“细枝末节”;并且,上述的“自助游”的风险和不确定性,也大多出自于此。鉴于此,我们认为,这其中存在着不合理性。而我们的这个应用,正是要去解决这不合理之处。

	\footnotetext{需要说明的是,很多时候,“食、住、行”三者本身也是旅行的意义所在。但这部分,我们将其归纳到前述的“旅行体验”中,此处所指的是除这些以外的部分。}

	% section 1.2
	\subsection{功能介绍}
	我们这个应用的目的,就是为用户订制完备及详细的“自助游”旅行方案,省去用户处理旅途细节的忧虑及烦恼。用户在使用这款应用,来规划自己的“自助游”方案时,只需要提交以下信息:
	\begin{itemize}
	\item 希望前往的地点(城市)
	\item 有兴趣的景点或体验项目
	\item 旅途时间,预算等少量其它信息
	\end{itemize}
	我们的应用程序就可以根据用户提交的要求,为用户制定一份完整且详细的“自助游”方案。该方案具体将包括以下内容:
	\begin{itemize}
	\item 旅行日程(哪一天参观哪些景点)
	\item 食宿安排(用餐地点,酒店的位置)
	\item 交通规划(交通方式及各段交通耗时)
	\item 注意事项等其他信息
	\end{itemize}
	并且将以顺序表或流程图的形式,直观地向用户展示我们的推荐方案。当然,用户拥有最终决定权,在我们呈现了推荐方案后,用户可以继续根据自己的需要来修改这份方案,直到自己满意,并做出最终的决定。方案被制定好以后,相关文件会在云端及用户本地分别保存,不仅增加了设备无关性,也防止在旅行途中由于上不了网而无法随时查看。
	
	% section 1.3
	\subsection{现有应用}
	必须承认,如今市面上已经有不少旅游相关的软件、手机应用及网站。对这些应用,我们可以大体上将它们划分为以下几个类别:
	有的提供旅游地点的信息,有的可以预定酒店,有的分享游记和旅行经历……

	% section 1.4
	\subsection{创新之处}
	上面列举了不少旅游相关的网站及应用。而总体而言,现有的应用,更多的是停留在信息传递方面,也就是说,仅仅是把大量的信息提供给用户。我们并没有发现一款软件或者应用,可以主动地为旅行者制定旅行方案提供直接的帮助。


\clearpage
\section{开发方案}

	% section 2.1
	\subsection{实现方案}
	在阐述我们的实现方案之前,有必要简要说明一下我们预想的这个应用的使用流程。在阐述完使用流程后,我们将结合使用流程,来讲解这个应用的具体实现方案。

	首先,这个应用大体的使用流程列举如下:
	\begin{enumerate}
		\item
		用户选定旅游目的地,具体到城市名,并且提供旅游预算,旅行时间等信息;
		\item
		应用展示该城市内及周边的景点,以及景点相关的信息,供用户挑选;
		\item
		根据用户的选择,为用户规划路线,安排行程及食宿,向用户呈现一份推荐方案;
		\item
		用户对方案进行修改后,保存。
	\end{enumerate}
	
	其次,按照我们的设想,我们的应用将会由两个部分组成:用户手机中的应用软件(客户端),以及架设在云端的服务器。其中,正如大多数的云应用一样,本地的客户端,即用户手机中的应用并不进行太多的运算和操作,仅仅是为用户提供一个图形界面,收集用户的输入信息,并将这些信息上传到云端,由云端进行处理。云端处理完毕得出结果以后,将其发回给本地的客户端。本地的客户端对传回的数据进行一定的图形渲染,并最终以图形化的形式呈现给用户。因此,结合上述的使用流程,具体的应用整体实现方式如下:
	\begin{description}
	\item[建立云端数据库] \hfill \\
	这是搭建本应用的第一个步骤。我们认识到,如今已经有不少旅游相关的网站和信息提供者。因此,我们不需要生成我们自己的数据,只需要将已有应用的数据收集并整合起来,例如借助百度地图的API搜集地理信息,从大众点评网收集对餐厅和酒店的评价,从旅游地的当地政府部门网站中收集景点及政策的相关信息等。云端的一些优点,例如24小时不关机,网络带宽大,廉价且弹性存储空间等,就可以在这里被运用:我们可以将“爬虫”程序部署在云端,以比较优良的网络带宽,连续不断地收集并更新云端的数据,然后将这些数据,放到云端的数据库管理系统(DBMS)中进行维护和管理。对于“爬虫”程序,网络上已经有不少现有的程序可供修改和使用,我们设想的是使用Python语言来实现,而DBMS则使用MySQL。
	\item[前期交互] \hfill \\
	在手机App中保存一份中国城市的列表,用户在UI界面中选定其中一个城市,并设置一些附加信息。用户输入完毕后,App将用户输入的信息上传到云端,云端根据一些内置数据及由其他地图应用获取的数据,收集选定城市周边的景点,将这些景点信息回传给用户。用户在手机App中选择好感兴趣的景点后,App向云端发送少量数据,指定这些景点。
	\item[决定景点的优先级] \hfill \\
	由于各个景点之间存在差异,为每个景点都分配同样多的时间,显然不是个好主意。因此云端先根据一套事先做好的,内置的标准,对这些景点做出区分并排序,排名靠前的景点,需要被安排比较多的时间,并且对它的行程要被安排得比较早。当然,在这个排序的过程中,用户对各个景点的偏好也应该被考虑。我们可以让用户在选择景点的时候顺便附上感兴趣的程度。
	\item[景点日程安排] \hfill \\
	包括各个景点的游览时间,以及前后景点之间的交通线路规划。在旅行的过程中,游客自然不希望在路途上花费大量的时间。这意味着,在安排日程的时候,我们需要对用户指定的景点按照它们的地理位置进行聚类,相互邻近的景点可以被连续地安排在一起,这样就可以减少用户耗费在路途上的时间。
	\item[食宿安排] \hfill \\
	安排好了景点的日程之后,就可以根据若干景点的地理位置,搜索其附近的餐饮及住宿服务。在搜索餐厅和旅馆的时候,也不需要对所有景点的周边都进行搜索,而只需要在那些安排于用餐时间前后游览的景点周边进行搜索。这么做,不仅减少了搜索量,降低了运算量,而且还有利于用户获得更人性化的旅游方案。这一步以及前两个步骤,我们计算使用C{\slash}C++程序,结合MySQL对C语言的API接口来实现。
	\item[整理并回传方案] \hfill \\
	至此,一份推荐给用户的方案就基本上在云端制作完成了。这份方案预计会以文本文件的形式被生成出来,云端需要将这份推荐方案回传到用户的手机App上,而手机上的App则读取这份文本文件,通过编排和渲染,以用户友好的方式将我们推荐的方案完整地呈现出来。数据回传的过程,我们预计使用云计算实验课上介绍的Python语言的webpy库,或者也可以用C语言程序创建一个Socket来实现。
	\item[用户修改并确认] \hfill \\
	用户可以细致地浏览这份方案,并且修改其中的任何细节。在用户修改完毕点击确认之后,如果原先的方案有被修改过,那么修改过后的方案文件会被上传到云端,同时也会在用户的手机中保留一份副本,方便用户快速地查看和浏览。
	\end{description}

	最后就是云平台的选择。综上所述,我们应用的实现,在云端要运用到Python,MySQL以及C{\slash}C++程序,并且还有一个用Java+XML实现的Android应用程序。可见,我们的云端必须是一台支持多种语言,而且功能相对完备的虚拟机。在选择云服务提供商的时候,我们必须选择那些可以在云上架设虚拟机的服务商。在云计算实验课上,TA们向我们展示了一些公司提供的云服务,其中最符合我们需求的,无疑是微软的Windows Azure。它提供的是一台配置可供选择的云端虚拟机,我们可以在上面做一切能够在本地PC上做的事情,例如运行操作系统,安装软件,配置开发环境等等。在Windows Azure上不仅可以选择虚拟机器的配置,还可以选择要安装的操作系统。根据我们的需求,我们认为Ubuntu 14.04LTS操作系统是我们的首选,因为它对上述三种语言都有很好的开发以及运行环境支持,并且对网络和内存的管理都比较好。
	
	% section 2.2
	\subsection{可行性分析}

	% section 2.3
	\subsection{关键技术问题}
	我们这个应用开发起来,还是有一定难度的,需要解决不少技术上的问题。下面将对已经考虑到的技术问题进行阐述。
	
	首先是收集网络上已有的信息,组成我们自己的云端数据库。云端是一台24小时运行的虚拟机,它可以不间断地运行,收集大量的信息。云端需要收集的信息大概包括:
	\begin{enumerate}
	\item	旅游网站提供的景点信息;
	\item	地图网站提供的地理坐标及周边环境信息;
	\item	点评类网站提供的客户评价信息;
	\item	景点官方网站或地方政府网站的法律及旅游政策信息。
	\end{enumerate}

	

	% section 2.4
	\subsection{开发计划及进度安排}



\end{CJK}
\end{document}

